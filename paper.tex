% This version uses the latex2e styles, not the very ancient 2.09 stuff.
\documentclass[letterpaper,twocolumn,10pt]{article}
\usepackage{usenix,epsfig,endnotes}
\usepackage[english]{babel}
\usepackage[bookmarks=true,pdfborder={0 0 0}]{hyperref}
\usepackage{url}
\usepackage[T1]{fontenc}
\usepackage[latin9]{inputenc}
\usepackage{color}
\usepackage{array}
\usepackage{textcomp}
\usepackage{multirow}
\usepackage{amsmath}
\usepackage{amsthm}
\usepackage{graphicx}
\usepackage{makecell}
\usepackage{enumitem}

\makeatletter
\providecommand{\tabularnewline}{\\}
\date{}

\renewcommand{\dblfloatpagefraction}{0.95}
\renewcommand{\floatpagefraction}{0.95}

\setlength{\belowcaptionskip}{-1ex}
\usepackage{listings}
\lstset{
    basicstyle=\small\ttfamily,
    tabsize=2,
    columns=fullflexible,
    keepspaces=true,
    language=c
}

\renewcommand{\paragraph}{%
  \@startsection{paragraph}{4}%
  {\z@}{1ex \@plus 1ex \@minus .2ex}{-1em}%
  {\normalfont\normalsize\bfseries}%
}

\makeatother
\usepackage{listings}
\renewcommand{\lstlistingname}{Listing}

% (jsun): this disables page numbers
\pagestyle{empty}

\begin{document}
\title{\Large \bf Breaking Apart the VFS for Managing File Systems}
\author{
  {\rm Kuei Sun, Matthew Lakier, Angela Demke Brown and Ashvin Goel}\\
  University of Toronto
} % end author

\maketitle

\begin{abstract}
File system management applications, such as data scrubbers, defragmentation tools, resizing tools, and partition editors, are essential for maintaining, optimizing, and administering storage systems. These applications require fine-grained control over file-system metadata and data, such as the ability to migrate a data block to another physical location. Such control is not available with the VFS API, and so these applications bypass the VFS and access and modify file-system metadata directly. As a result, these applications do not work across file systems, and must be developed from scratch for each file system, which involves significant engineering effort and impedes adoption of new file systems.

Our goal is to design an interface that allows these management applications to be written once and be usable for all file systems that support the interface. Our key insight is that these applications operate on common file system abstractions, such as file system objects (e.g., blocks, inodes, and directory entries), and the mappings from logical blocks of a file to their physical locations. We propose the Extended Virtual File System (eVFS) interface that provides fine-grained access to these abstractions, allowing the development of generic file system management applications. We demonstrate the benefits of our approach by building a file-system agnostic conversion tool that performs in-place conversion of a source file system to a completely different destination file system, showing that arbitrary modifications to the file system format can be handled by the interface.

\end{abstract}

\begin{table*}
\begin{center}
\begin{small}
\begin{tabular}[t]{|l|l|}
\hline 
\textbf{Function Prototype} & \textbf{Description} \\
\hline
\hline
\texttt{struct evfs * fs\_open(struct evfs\_mount * mnt)} & open the file system with parameters specified by \textit{mnt} \\
\hline
\hline
\texttt{struct evfs\_txn * txn\_begin(struct evfs * evfs)} & start a new transaction and return the associated handle \\
\hline 
\texttt{int txn\_commit(struct evfs\_txn * txn)} & commit the transaction \textit{txn} \\
\hline 
\texttt{int txn\_abort(struct evfs\_txn * txn)} & abort the transaction \textit{txn} \\
\hline
\hline
\texttt{int super\_make(struct evfs\_super * sup)} & make new a file system with parameters specified by \textit{sup} \\
\hline
%\texttt{int super\_set(struct evfs\_super * sup)} & update an existing file system's settings \\
%\hline
\hline 
\texttt{s64 extent\_alloc(u64 addr, u64 len)} & allocate the extent defined by \{\textit{addr}, \textit{len}\} \\
\hline 
\texttt{long extent\_free(long addr, long len)} & free the extent defined by \{\textit{addr}, \textit{len}\} \\
\hline 
% (jsun): removed because not discussed or motivated in approach section
%\texttt{long extent\_write(long addr, long len, char * data)} & write \textit{data} to the extent defined by \{\textit{addr}, \textit{len}\} \\
%\hline 
%\texttt{long extent\_read(long addr, long len, char * data)} & read \textit{data} from the extent defined by \{\textit{addr}, \textit{len}\} \\
%\hline 
\makecell[l]{\texttt{s64 extent\_reverse(u64 addr, u64 len,} \\
\hspace{1em}\texttt{struct evfs\_reverse * rv)}} & \makecell[l]{fills \textit{rv} with the inode number and logical offset of \\ all inodes that map to the extent defined by \{\textit{addr}, \textit{len}\}} \\
\hline 
\texttt{int extent\_active(u64 addr, u64 len)} & return 1 if extent defined by \{\textit{addr}, \textit{len}\} is active, else 0 \\
\hline 
\makecell[l]{\texttt{s64 extent\_iterate(s64 ino\_nr, void * priv,} \\ 
\hspace{1em}\texttt{s64 (* cb)(u64 log\_blk\_nr, u64 phy\_blk\_nr,} \\
\hspace{6.75em}\texttt{u64 len, void * priv))}} & \makecell[l]{iterate through all extents mapped to inode \textit{ino\_nr} in \\ the form of \{\textit{log\_blk\_nr}, \textit{len}\} $\rightarrow$ \{\textit{phy\_blk\_nr}, \textit{len}\} \\ and process them via callback function \textit{cb}} \\
\hline 
\makecell[l]{\texttt{s64 freesp\_iterate(void * priv, s64 (* cb)(} \\ 
\hspace{1em}\texttt{u64 addr, u64 len, void * priv))}} & \makecell[l]{iterate through all free space extents in the file system \\ and process them via callback function \textit{cb}} \\
\hline
\hline
\texttt{s64 inode\_alloc(s64 ino\_nr, struct evfs\_inode * i)} & allocate the inode \textit{ino\_nr} with the inode structure \textit{i} \\
\hline 
\texttt{long inode\_free(long ino\_nr)} & free the inode \textit{ino\_nr} \\
\hline 
\makecell[l]{\texttt{s64 inode\_read(s64 ino\_nr, s64 ofs, char * data,} \\
\hspace{7.85em}\texttt{u64 len)}} & \makecell[l]{read \textit{len} byte of data to \textit{data}
 from the inode \textit{ino\_nr} \\ at logical offset \textit{ofs}} \\
\hline 
% (jsun): removed because not significant
%\makecell[l]{\texttt{long inode\_write(long ino\_nr, long ofs, char * data,} \\
%\hspace{1em}\texttt{long len)}} & \makecell[l]{write \textit{data} of \textit{len} byte to the inode \textit{ino\_nr} at logical \\ offset \textit{ofs}} \\
%\hline 
%\texttt{long inode\_set(long ino\_nr, struct evfs\_inode * i)} & set the inode \textit{ino\_nr} with the inode structure \textit{i} \\
%\hline
\makecell[l]{\texttt{int inode\_map(u64 ino\_nr, u64 log\_blk\_nr,} \\
\hspace{7.35em}\texttt{u64 phy\_blk\_nr, u64 len)}} & \makecell[l]{map physical extent \{\textit{phy\_blk\_nr}, \textit{len}\} to the logical \\ extent \{\textit{log\_blk\_nr}, \textit{len}\} for inode \textit{ino\_nr}} \\
\hline 
%\makecell[l]{\texttt{int inode\_unmap(s64 ino\_nr, u64 addr, u64 len)}} & \makecell[l]{unmap logical extent \{\textit{addr}, \textit{len}\} for inode \textit{ino\_nr}} \\
%\hline
% (jsun): removed because not discussed or motivated in approach section
%\texttt{long inode\_truncate(long ino\_nr, long nblks)} & reduce the number of used blocks for %\textit{ino\_nr} to \textit{nblks} \\
%\hline 
%\texttt{int inode\_active(long ino\_nr)} & returns 1 if inode \textit{ino\_nr} is active, else 0 \\
%\hline 
\makecell[l]{\texttt{s64 inode\_iterate(void * priv, s64 (* cb)(} \\ 
\hspace{1em}\texttt{s64 ino\_nr, struct evfs\_inode * i, void * priv))}} & \makecell[l]{iterate through all active inodes in the file system and \\ process them via callback function \textit{cb}} \\
\hline
\hline
\texttt{int dirent\_add(s64 dir\_nr, struct evfs\_dirent * d)} & add a new entry \textit{d} to directory inode \textit{dir\_nr} \\
\hline 
%\texttt{long dirent\_remove(long dir\_nr, long ino\_nr)} & remove entry \textit{ino\_nr} from directory inode \textit{dir\_nr} \\
%\hline 
% (jsun): removed because not significant
%\texttt{long dirent\_active(long dir\_nr, long ino\_nr)} & returns 1 if entry \textit{ino\_nr} exists in \textit{dir\_nr}, else 0 \\
%\hline 
\makecell[l]{\texttt{s64 dirent\_iterate(s64 dir\_nr, void * priv,} \\ 
\hspace{1em}\texttt{s64 (* cb)(struct evfs\_dirent * d, void * priv))}} & \makecell[l]{iterate through all directory entries for inode \textit{dir\_nr} \\ and process them via callback function \textit{cb}} \\
\hline
\end{tabular}
\end{small}
\end{center}
\vspace{-16pt}
\caption{\label{tab:evfs-api}eVFS API. Parameter \texttt{struct evfs\_txn} is omitted for all functions except for the first four functions.}
\vspace{-6pt}
\end{table*}

\vspace{-0.7em}
\section{Introduction\label{sec:Introduction}}
\vspace{-0.4em}

File system management applications help with maintaining, optimizing, and administering file systems. Examples of such applications include file system upgrade tools, defragmentation tools, and file system resizing tools. Unlike typical applications that are file-system agnostic because they use the virtual file-system interface (VFS) to access their data, the management applications perform low-level allocation, mapping, and placement of physical blocks in a file system. These operations are not exposed by the VFS API, and thus these applications must bypass the VFS, and access the file system metadata directly.

As a result, file system developers and experts must write these applications from scratch for each file system, because they are tightly coupled with the format of the file system. For example, a defragmentation tool for Ext4 cannot be reused for Btrfs, not even in parts, because the two file systems use different formats for block allocation and free space management. The effort required for building these applications is significant, and thus newer file systems such as F2FS~\cite{lee2015f2fs} and BetrFS~\cite{betrfs-2015-tos} lack a rich set of management tools, which stymies their adoption and hinders innovation in file system technology.

The goal of our work is to simplify the development of file system management applications. The VFS interface has been highly successful because it abstracts the key objects (e.g., files and directories) and operations (e.g., create, delete, read, write) that are provided by any file system. Our approach is to provide a new abstraction, similar to VFS, that enables file system management applications to be written in a generic, file-system agnostic manner. Ideally, the applications are developed once using this interface and they work for file systems that implement this interface.  This relieves file system developers from the onus of building these essential (but often neglected) applications, and instead they can focus their effort on improving the file system itself.

We introduce the \textbf{E}xtended \textbf{V}irtual \textbf{F}ile \textbf{S}ystem (eVFS) interface, which provides a fine-grained abstraction for manipulating the file system. The key insight of eVFS is that the management applications operate on common abstractions that are shared across file systems, such as the allocation of file system objects (e.g., blocks or extents, inodes, and directory entries) and the mappings from logical blocks of a file to physical blocks. By exposing these abstractions, the eVFS API enables building applications that work across file systems. For example, a defragmentation tool needs to find the fragmented blocks of a file and relocate them to a contiguous extent. It can do so generically by making an extent allocation and using the logical to physical extent remapping operations provided by the API.

The eVFS API does not change the file system's trust model. Management applications are already trusted to operate directly on metadata without the VFS, and bugs in them may cause file system inconsistency or corruption~\cite{Carreira2012,Gunawi08b}. Hence, exposing these operations through eVFS may improve the robustness of management applications, since the file-system specific implementation of the interface can be provided once by file system experts.

Eventually, our aim is to expose these operations for online use without affecting existing file system applications that are unaware of management applications or the eVFS API. To do so, the eVFS API provides a transactional interface for eVFS operations. Currently, however, we have only explored offline use, where the transactional support provides crash consistency, which is often missing in management applications~\cite{gatla2018fsck}.

%The eVFS interface enables building applications that can be used across file systems. Consider a defragmentation tool that aims to reduce fragmentation in the file system. This application needs to find fragmented files and then relocate the fragmented blocks of the files into contiguous extents. Conventional defragmentation tools are coupled with the format of the file system. With the eVFS API, applications no longer need to operate on the low-level format of the file system. Instead, the API provides generic operations for allocating contiguous extents and remapping logical blocks of a file to a new extent. The decoupling of these ``subatomic'' file system operations from file system formats greatly simplifies the process of building these applications.

As proof of concept, we have built an offline file system conversion tool using the eVFS interface. This tool performs crash-consistent, in-place conversion of one file system to another entirely different file system. It can thus also be used to modify the file-system specific options of the file system, such as the file system size, or upgrade a file system. The application is generic, and thus supporting additional file systems should require no modifications to the application. This experience suggests that the eVFS API will allow building a variety of generic, file system management applications.

%% Ashvin - To do so, this interface must provide precise control over the allocation and placement of file system resources, such as blocks and inodes. Fortunately, file systems internally operate on these same resources, but VFS currently does not expose these operations.

%These operations already exist for file systems that implements VFS. However, they are not currently exposed because they can lead to inconsistency or resource leak if used incorrectly. Therefore, we break down the VFS operations into components which form the building blocks for file system managemenet applications, and provide a transactional interface to ensure consistency of eVFS operations.


\begin{table*}
\begin{center}
\begin{small}
\begin{tabular}[t]{|l|l|}
\hline 
\textbf{Function Prototype} & \textbf{Description} \\
\hline
\hline
\texttt{struct evfs * fs\_open(struct evfs\_mount * mnt)} & open the file system with parameters specified by \textit{mnt} \\
\hline
\hline
\texttt{int super\_make(struct evfs\_super * sup)} & make new a file system with parameters specified by \textit{sup} \\
\hline
\texttt{int super\_set(struct evfs\_super * sup)} & update an existing file system's settings \\
\hline
\hline 
\texttt{s64 extent\_alloc(u64 addr, u64 len)} & allocate the extent defined by \{\textit{addr}, \textit{len}\} \\
\hline 
%\texttt{long extent\_free(long addr, long len)} & free the extent defined by \{\textit{addr}, \textit{len}\} \\
%\hline 
% (jsun): removed because not discussed or motivated in approach section
%\texttt{long extent\_write(long addr, long len, char * data)} & write \textit{data} to the extent defined by \{\textit{addr}, \textit{len}\} \\
%\hline 
%\texttt{long extent\_read(long addr, long len, char * data)} & read \textit{data} from the extent defined by \{\textit{addr}, \textit{len}\} \\
%\hline 
\makecell[l]{\texttt{s64 extent\_reverse(u64 addr, u64 len,} \\
\hspace{1em}\texttt{struct evfs\_reverse * rv)}} & \makecell[l]{fills \textit{rv} with the inode number and logical offset of \\ all inodes that map to the extent defined by \{\textit{addr}, \textit{len}\}} \\
\hline 
\texttt{int extent\_active(u64 addr, u64 len)} & return 1 if extent defined by \{\textit{addr}, \textit{len}\} is active, else 0 \\
\hline 
\makecell[l]{\texttt{s64 extent\_iterate(s64 ino\_nr, void * priv,} \\ 
\hspace{1em}\texttt{s64 (* cb)(u64 log\_blk\_nr, u64 phy\_blk\_nr,} \\
\hspace{6.25em}\texttt{u64 len, void * priv))}} & \makecell[l]{iterate through all extents mapped to inode \textit{ino\_nr} in \\ the form of \{\textit{log\_blk\_nr}, \textit{len}\} $\rightarrow$ \{\textit{phy\_blk\_nr}, \textit{len}\} \\ and process them via callback function \textit{cb}} \\
\hline 
\makecell[l]{\texttt{s64 freesp\_iterate(void * priv, s64 (* cb)(} \\ 
\hspace{1em}\texttt{u64 addr, u64 len, void * priv))}} & \makecell[l]{iterate through all free space extents in the file system \\ and process them via callback function \textit{cb}} \\
\hline
\hline
\texttt{s64 inode\_alloc(s64 ino\_nr, struct evfs\_inode * i)} & allocate the inode \textit{ino\_nr} with the inode structure \textit{i} \\
\hline 
%\texttt{long inode\_free(long ino\_nr)} & free the inode \textit{ino\_nr} \\
%\hline 
\makecell[l]{\texttt{s64 inode\_read(s64 ino\_nr, s64 ofs, char * data,} \\
\hspace{7.85em}\texttt{u64 len)}} & \makecell[l]{read \textit{len} byte of data to \textit{data}
 from the inode \textit{ino\_nr} \\ at logical offset \textit{ofs}} \\
\hline 
% (jsun): removed because not significant
%\makecell[l]{\texttt{long inode\_write(long ino\_nr, long ofs, char * data,} \\
%\hspace{1em}\texttt{long len)}} & \makecell[l]{write \textit{data} of \textit{len} byte to the inode \textit{ino\_nr} at logical \\ offset \textit{ofs}} \\
%\hline 
%\texttt{long inode\_set(long ino\_nr, struct evfs\_inode * i)} & set the inode \textit{ino\_nr} with the inode structure \textit{i} \\
%\hline
\makecell[l]{\texttt{int inode\_map(u64 ino\_nr, u64 log\_blk\_nr,} \\
\hspace{7.3em}\texttt{u64 phy\_blk\_nr, u64 len)}} & \makecell[l]{map physical extent \{\textit{phy\_blk\_nr}, \textit{len}\} to the logical \\ extent \{\textit{log\_blk\_nr}, \textit{len}\} for inode \textit{ino\_nr}} \\
\hline 
\makecell[l]{\texttt{int inode\_unmap(s64 ino\_nr, u64 addr, u64 len)}} & \makecell[l]{unmap logical extent \{\textit{addr}, \textit{len}\} for inode \textit{ino\_nr}} \\
\hline
% (jsun): removed because not discussed or motivated in approach section
%\texttt{long inode\_truncate(long ino\_nr, long nblks)} & reduce the number of used blocks for %\textit{ino\_nr} to \textit{nblks} \\
%\hline 
%\texttt{int inode\_active(long ino\_nr)} & returns 1 if inode \textit{ino\_nr} is active, else 0 \\
%\hline 
\makecell[l]{\texttt{s64 inode\_iterate(void * priv, s64 (* cb)(} \\ 
\hspace{1em}\texttt{s64 ino\_nr, struct evfs\_inode * i, void * priv))}} & \makecell[l]{iterate through all active inodes in the file system and \\ process them via callback function \textit{cb}} \\
\hline
\hline
\texttt{int dirent\_add(s64 dir\_nr, struct evfs\_dirent * d)} & add a new entry \textit{d} to directory inode \textit{dir\_nr} \\
\hline 
%\texttt{long dirent\_remove(long dir\_nr, long ino\_nr)} & remove entry \textit{ino\_nr} from directory inode \textit{dir\_nr} \\
%\hline 
% (jsun): removed because not significant
%\texttt{long dirent\_active(long dir\_nr, long ino\_nr)} & returns 1 if entry \textit{ino\_nr} exists in \textit{dir\_nr}, else 0 \\
%\hline 
\makecell[l]{\texttt{s64 dirent\_iterate(s64 dir\_nr, void * priv,} \\ 
\hspace{1em}\texttt{s64 (* cb)(struct evfs\_dirent * d, void * priv))}} & \makecell[l]{iterate through all directory entries for inode \textit{dir\_nr} \\ and process them via callback function \textit{cb}} \\
\hline
\hline
\texttt{struct evfs\_txn * txn\_begin(struct evfs * evfs)} & start a new transaction and return the associated handle \\
\hline 
\texttt{int txn\_commit(struct evfs\_txn * txn)} & commit the transaction \textit{txn} \\
\hline 
\texttt{int txn\_abort(struct evfs\_txn * txn)} & abort the transaction \textit{txn} \\
\hline
\end{tabular}
\end{small}
\par\end{center}
\vspace{-15pt}
\caption{\label{tab:evfs-api}eVFS API. Parameter \texttt{struct evfs\_txn} is omitted for all functions except for \texttt{fs\_open} and the last 2 transaction-related functions. All Functions return negative error code upon failure.}
\vspace{-6pt}
\end{table*}

\vspace{-0.5em}
\section{Approach\label{sec:Approach}}

The goal of designing the eVFS interface is to enable file-system agnostic management applications. As such, the interface must be generic while providing fine-grained control over the allocation of file system objects, and mappings from one object to another (e.g. directory entries to files, files to blocks). Therefore, we must first define the various file system objects that are generic across file systems.

At a high level, file systems manage four types of objects: files or directories, blocks or extents, directory entries, and file-system wide settings (such as the block size, file system size, or label). Thus, we provide an interface for managing each of the objects, and any mappings between them. In this section, we motivate the eVFS design by describing use cases for accessing and manipulating these objects, including the need for transactional support.

\paragraph{Inodes} In the eVFS interface, similar to VFS, every file system object, such as a file or directory, is uniquely identified by an inode number and structure. File system management applications frequently need to read, create or update inode structures and their mappings to physical blocks.
For example, a defragmentation tool needs to scatter-gather fragmented blocks of a file into a new contiguous extent, which involves updating the logical to physical block mappings of an inode. The eVFS interface thus provides support for allocating and updating an inode. The inode allocation interface is finer-grained than VFS file creation, since it does not allocate a directory entry or file blocks. The eVFS interface also allows mapping and unmapping logical offsets of a file to specific physical blocks, providing precise control over these mappings.

\paragraph{Blocks and Extents} Many file system management applications require fine-grained control over the physical layout of a file on disk. For example, an in-place file system conversion tool needs to recreate files on the destination file system that directly map to existing data blocks belonging to the same files on the source file system, while avoiding copying blocks as much as possible (see Section~\ref{subsec:conversion_tool} for more detail). Thus, the interface allows allocation of blocks and extents at specific physical addresses.

To allocate blocks and extents, management applications need to know the locations and sizes of free spaces. Maintenance applications also require knowledge of the remaining free space in the file system to determine whether to start garbage collection. This information must be obtained by processing block allocation metadata and is thus a file-system specific operation. However, with eVFS, we abstract away the file-system specific details and provide a function that enables applications to iterate through all free space extents in a file system, or find the nearest available free extent, without needing to know the format of the file system. Similarly, the API allows applications to check whether an extent is currently in use.

% This function helps the file system conversion tool detect when an extent must be copied elsewhere due to conflict.

% (Ashvin): clarify the reason for the skip. given the reverse interface, is it always possible to move extents? clarify. the power of the reverse interface is not clear right now. In particular, for copy-on-write and log-structured file systems, moving a block might require updates to the parent, which requires progressively updating the parent pointers, etc. do we simplify that? support that?

For file systems that support copy-on-write semantics and snapshots, management applications can make informed decisions based on whether an extent is private to a file or shared by multiple files. For example, it is easier to relocate private extents during garbage collection. To enable such logic, the interface supports retrieving a reverse mapping that lists the inodes and their logical offsets that map to a particular extent. With this information, a defragmentation tool can move an extent by remapping all inodes that reference the extent to its new location.

\paragraph{Directory Entries} File systems use directory entries to support mapping name(s) to an inode. Consider our in-place file system conversion tool that needs to recreate directories. It must iterate through the entries in the source file system while making copies to the destination file system. Therefore, the interface supports adding, updating, or removing individual directory entries, as well as iterating through the entries of a directory inode.

\paragraph{File-System Wide Settings} A file system stores various parameters and options that describe the file system format and the features that are supported. Some of these parameters are common across different file systems, such as the total size of the file system, block size, etc. Therefore, they can be exposed to support management applications that modify the layout or format of the file system (e.g., updating file system to a newer version, changing the block size of an existing file system).

The eVFS interface provides generic support for managing file-system wide settings in two ways. First, it allows updating simple settings such as labels or file system feature flags that do not require restructuring the file system, similar to the functionality provided by \texttt{tune2fs}~\cite{tso-e2fsprogs} for the Ext3/Ext4 file systems. Second, to support generic restructuring, the interface provides support for creating an empty file system, that performs the same task as \texttt{mkfs}. As described later in more detail, this interface allows our in-place file system conversion tool to reformat the device to the destination file system, and then the new file system metadata can be recreated, while keeping the file contents of the existing file system intact. Similarly, a file system can be resized using this approach, although less efficiently than a custom file-system specific resizing tool.

\paragraph{Transactions} Since many of the operations supported by the interface make the file system temporarily inconsistent, the interface also provides transactional support to ensure atomicity so that other applications do not see partial updates made by management applications. Transactional support is also necessary for providing crash consistency, which is often missing in management applications~\cite{gatla2018fsck}. Thus, eVFS also enables building robust management applications that are resilient to power failures.




\vspace{-0.5em}
\section{Implementation\label{sec:Implementation}}

In this section, we present our prototype of the eVFS API and discuss our implementation of the API. Next, we describe the in-place file system conversion tool that we have built using the eVFS interface.

Table~\ref{tab:evfs-api} shows a partial set of functions in the eVFS API. These functions provide fine-grained control by allowing extents, inodes, and directory entries, to be individually manipulated. We chose to use extent-based representation for storage space since it generally requires less metadata than the corresponding block-based representation, and is thus preferred by modern file systems. An application is expected to start a transaction before issuing most eVFS operations. 

%% Ashvin - this is not saying anything too interesting: An extent is written in the form of \{\textit{addr}, \textit{len}\}, where \textit{addr} is the block address of the extent, and \textit{len} is the number of blocks that the extent occupies.

We have implemented a subset of the eVFS API for two file systems, the extent-based Ext4 file system, and the log-structured F2FS file system, which enables converting an Ext4 file system to an F2FS file system. Our current implementation only works for offline use, i.e., the application has exclusive access to the unmounted file system. The file-system specific implementation of the API uses the Spiffy framework~\cite{sun2018spiffy} that provides robust parsing and serialization libraries, helping avoid bugs while handling file system metadata.

\vspace{-0.5em}
\subsection{File System Conversion Tool\label{subsec:conversion_tool}}

Converting an existing file system to a different file system is a tedious and time-consuming process that normally involves copying the full content of a file system to another disk, reformatting the disk, and then copying everything back to the new file system. In contrast, an in-place file system conversion only updates file system metadata, and does not move any regular file data unless its location must be used by statically allocated metadata of the destination file system. This technique can speed up the conversion dramatically. While some such conversion tools exist, they are hard to implement correctly\footnote{e.g. the \texttt{convertfs} utility \cite{convertfs} requires sparse file support on the source file system and performs full data copying} and not generally available.

We have designed and implemented a crash-consistent, in-place file system conversion tool using the eVFS interface. To do so, the conversion tool uses user-level, block-based redo journaling for ensuring crash consistency. Unlike typical journals that have a fixed size (e.g., the Ext4 journal), the journal is dynamically allocated from blocks that are currently free in both the source and the destination file systems, which ensures that both abort and redo recovery are possible since these blocks are not in use by either file system. The free space information is obtained by using the eVFS API. The dynamic allocation of blocks also allows converting heavily fragmented file systems, and maximizing utilization of the free space for the journal.

As an optimization, when a destination file system block is written to free space in the source file system, the block is written directly without being journaled. By journaling the rest of the blocks that will overwrite the source file system, we ensure crash consistency.

A complication occurs when a block that is currently in use by the journal is allocated to the destination file system. Allocating this block would cause the journal to be overwritten during checkpointing. In this case, the conflicting journal block is remapped to a different free block, and then this freed block can be updated directly.

If the journaling layer runs out of free space, the conversion process is aborted. We guarantee that this error occurs before the transaction commits. Therefore, no data loss is possible on a conversion failure. This guarantee enables streaming metadata read from the source file system while writing metadata for the destination file system. Without journaling, the metadata of the file system may be overwritten and thus all metadata must be fully read into memory before commencing the conversion. Therefore, journaling also helps reduce memory overhead, which is essential for converting large file system images.
 
The conversion tool starts a new transaction and then creates an empty destination file system on the device storing the source file system. Next it iterates through the inodes of the source file system, and creates the corresponding inodes in the destination file system. For a regular file, it iterates through each extent in the source inode, allocating the corresponding extent in the destination file system, and then copying over the mappings to the destination inode. For a source file system extent, we also check whether it overlaps with block(s) that are allocated in the destination file system. If so, we relocate the extent to free space in the destination file system, and update the source inode that maps to this extent. For directories, we iterate through the entries to recreate them in the destination file system. Then we commit the transaction and allow checkpointing to create the destination file system. The commit information needs to be located in a well-known location that is not in use by either file system. Currently, we use the boot record to store this information.

File systems sometimes inline data inside metadata (e.g., Ext4's fast symlink) or perform subblock allocation (e.g., ReiserFS's tail packing). The conversion tool can detect these cases through the eVFS interface, and it would instead read the content of the inode through a function similiar to the regular VFS \texttt{read} and rebuild the content using the equivalent of VFS \texttt{write} for the destination file system.

When converting to a file system that lacks the feature of the source file system, suitable defaults are used if possible, but some information may be lost as a result. For example, converting a file system with immutable snapshots to Ext4 will result in a copy of the snapshots being created, since Ext4 does not support snapshots. Similarly, converting a file system with quota support to one that does not will result in the loss of quota-related metadata.

%Data relocation involves reading the extent from the source file system and writing it to the destination file system.




\vspace{-0.75em}
\section{Evaluation\label{sec:Evaluation}}

In this section, we evaluate the programming effort needed to build the in-place file system conversion tool. The file system conversion tool is based on our previous work~\cite{sun2018spiffy}, and redesigning the application to use the eVFS interface did not result in statistically significant changes to its performance, and so we omit the discussion of our performance results.

%\subsection{Programming Effort}

\begin{table}
\begin{centering}

\begin{tabular}{|llc|llc|}
\hline 
\multicolumn{3}{|c|}{\textbf{Spiffy Converter}} & \multicolumn{3}{c|}{\textbf{eVFS Converter}}\tabularnewline
\hline 
\multicolumn{3}{|l|}{Application} & \multicolumn{3}{l|}{Application}\tabularnewline
~ & Generic & 504 & ~ & Generic & 224\tabularnewline
 & Ext4 & 218 &  & Ext4 & -\tabularnewline
 & F2FS & 1780 &  & F2FS & -\tabularnewline
\hline 
\multicolumn{3}{|l|}{Libraries} & \multicolumn{3}{l|}{Libraries}\tabularnewline
 & Generic & 2250 &  & Generic & 2625\tabularnewline
 & Journaling & - &  & Journaling & 1350\tabularnewline
 & Ext4 & - &  & Ext4 & 276\tabularnewline
 & F2FS & - &  & F2FS & 2152\tabularnewline
\hline 
\end{tabular}

\par\end{centering}
\vspace{-5pt}
\caption{\label{tab:programming-effort}Lines of code for the Spiffy and the eVFS file-system conversion tools.}
\vspace{-5pt}
\end{table}

Table~\ref{tab:programming-effort} shows the programming effort for building the file-system conversion tool using the Spiffy framework~\cite{sun2018spiffy} (Spiffy converter) and the eVFS interface (eVFS converter). Both the converters use the same logic, but the Spiffy converter's application code uses 2502 lines, which includes almost 2000 lines of file-system specific code, and this converter can only convert from Ext4 to F2FS. The eVFS converter uses 224 lines of generic file-system conversion code, less than 10\% of the Spiffy application), and could be used to convert between any pair of file systems that implement the eVFS API. The libraries used by both applications provide generic code (e.g., bitmaps, hash tables, etc.) for supporting management applications. The file-system specific code used by the eVFS converter is part of the eVFS library and can be used by other management applications. Unlike the Spiffy converter, the eVFS converter is crash-consistent, requiring 1350 lines of journaling code.
  
\iffalse
\subsection{Performance}

\begin{table}
\begin{centering}
\begin{tabular}{|c||c|c|}
\hline 
\# of files & Copy Converter & eVFS Converter \\
\hline
\hline 
20000 & $188.17 \pm 3.65$s & $7.03 \pm 0.2$s  \\
\hline 
5000 & $190.28 \pm 2.15$s & $4.01 \pm 0.09$s \\
\hline 
1000 & $192.74 \pm 2.28$s & $3.84 \pm 0.03$s \\
\hline
100 & $195.11 \pm 0.18$s & $3.71 \pm 0.13$s \\
\hline 
\end{tabular}
\par\end{centering}
\vspace{-5pt}
\caption{\label{tab:fsconvert-result}Time required for each technique to convert 
from Ext4 to F2FS for different number of files.}
\vspace{-5pt}
\end{table}

We compare the time it takes to perform copy-based conversion, versus using our eVFS based file-system conversion tools. The results are shown in Table~\ref{tab:fsconvert-result}. The experiments are run on an Intel 510 Series SATA SSD.  We create the file set using Filebench 1.5-a3~\cite{wilson2008new} in an Ext4 partition on the SSD, and then convert the partition to F2FS. The 20K file set uses the~\texttt{msnfs} file size distribution with the largest file size up to 1GB. The rest of the file sets have progressively fewer small files. All file sets have a total size of 16GB. For the copy converter, we run \texttt{tar -aR} at the root of the SSD partition and save the tar file on a separate local disk. We then reformat the SSD partition and extract the file set back into the partition.

The copy converter requires transferring two full copies of the file set, and so it takes 30x to 50x longer than using the conversion tool, which only needs to move data blocks out of locations that are used by the destination file system, and then create just the metadata for the destination file system. The conversion tool takes more time with larger file sets since it needs to convert more file system metadata. Overall, we show that applications written using the eVFS interface can achieve good performance.
\fi


\vspace{-0.75em}
\section{Related Work\label{sec:Related_Work}}

We had previously built a file system conversion tool using the Spiffy framework~\cite{sun2018spiffy}. Spiffy uses an annotation language to enable complete specification of the file system format and then generates a robust library for parsing and serializing the file system data structures. However, Spiffy only helps identify the types of these structures, and not their semantics, and thus still requires significant file-system specific code. In contrast, the eVFS interface is generic across file systems. In this work, we use Spiffy for the file-system specific implementation of the eVFS API, thus ensuring a robust implementation.

%We had previously built a file system conversion tool using the Spiffy framework~\cite{sun2018spiffy}. Spiffy uses an annotation language to enable complete specification of the file system format, and uses annotated file system structure definitions to generate a library. This library provides parsing and serialization routines, and thus helps simplify processing file systems metadata via generic traversal functions, and building more robust applications. However, applications written using Spiffy are still file-system specific, since Spiffy only helps identify the types of file system structures, and not the semantics of the structures and their fields. Furthermore, applications that require initializing and updating file-system specific structures require significant file-system specific code. Therefore, writing Spiffy applications still requires significant effort. In contrast, the eVFS interface is truly generic across file systems; applications can be written in a fully file-system agnostic manner. In this work, we use Spiffy for the file-system specific implementation of the eVFS API, thus ensuring a robust implementation.

There are several libraries for accessing and manipulating file systems, such as \texttt{libext2fs}~\cite{tso-e2fsprogs} and \texttt{libfsntfs} that comes with \texttt{ntfsprogs}~\cite{mathes2007ntfs}. While most of the functions provided by these libraries are file-system specific, some are generic across file systems, such as iterating through all inodes in the file system, which we have adapted for the eVFS interface.

The \texttt{DeviceIoControl} function in the Win32 API supports many control codes that enable fine-grained changes to the file system and its resources. For example, \texttt{FSCTL\_MOVE\_FILE} allows for atomic remapping of a file's blocks~\cite{win32-defrag}. The Win32 file and volume management API ensures that each operation results in a consistent file system state. In contrast, most eVFS operations can cause resource leaks or inconsistency if used incorrect. However, eVFS enables more powerful file system management applications, such as the conversion tool. 

Many parallel file systems use a dedicated metadata service backed by an object storage service that provides persistence for both file data and file system metadata~\cite{schwan2003lustre}. With namespace management decoupled from the data path, clients can stream data to the storage servers, thus maximizing I/O bandwidth~\cite{weil2006ceph}. While these object stores conceptually split the VFS API across metadata and data operations, the eVFS API splits the operations themselves.

%Many existing works extended storage interfaces to simplify writing file systems, while improving their reliability, security, durability, functionality, or performance. A type-safe disk~\cite{Sivathanu06} extends the disk interface by exposing primitives for block allocation and pointer relationships, which helps enforce invariants such as preventing access to unallocated blocks.  Nameless writes~\cite{zhang2012nameless} allow the device to control block allocation decisions, which improves garbage collection and wear leveling of solid-state drives. Isotope~\cite{shin2016isotope} is a block-level store that supports ACID transactions over block reads and writes, simplifying the design of higher-level transactional applications, or supporting applications that need to use different systems, such as a file system and a database. Range writes~\cite{anand08rangewrites} allow the file system to specify a set of addresses at which to write a block. The disk can then make the optimal choice and return the address it picked.

Many existing works extended storage interfaces to simplify writing file systems, while improving their reliability~\cite{Sivathanu06}, security, functionality~\cite{shin2016isotope}, or performance~\cite{zhang2012nameless,anand08rangewrites}. Our work is similar in spirit to these works in that we have extended the file system interface to reduce the effort of building file system management applications. However, instead of pushing the decision making down to the lower layer, we instead expose previously internal operations in the file system, thus enabling applications that require more fine-grained control over the file system.

\vspace{-1ex}

\vspace{-0.7em}
\section{Limitations and Future Work\label{sec:Future_Work}}
\vspace{-0.4em}

While our approach helps with building generic management applications, these applications will need to specifically handle file systems that support subsets of the eVFS API. For example, in-place update file systems, such as Ext4, generally do not track the reverse mapping from extents to inodes, and so cannot implement this API call efficiently. As a result, certain applications will either not be able to support these file systems, or will require different logic for such file systems.

We are currently working on supporting several deployed VFS-based file systems, such as Btrfs and XFS. Since our API is generic, we believe it should be possible to extend it for non-VFS file systems as well.

Our API is designed to provide control over extents, but these extents may be mapped to a non-linear physical address space. For example, modern file systems such as Btrfs and ZFS incorporate volume management and RAID-style redundancy within the file system, and thus the extents may map to physically discontiguous chunks of physical storage. Since some management applications may need control over these physical chunks as well, we plan to explore the feasibility of generically exposing this address space.

Our eVFS API is extent-based, and so we believe that it should be possible for new non-volatile memory (NVM) file systems to implement the API, allowing them to benefit from management applications originally designed for traditional block-based storage.

Our current eVFS implementation is designed for offline applications, and provides crash consistency support. For online applications, we plan to provide transactional support for eVFS operations. While a transactional file system would simplify this implementation, such file systems are not in common use because supporting transactions adds significant complexity to the entire kernel~\cite{Spillane2009}. Instead, we want to make minimal changes to existing file systems while ensuring that existing applications cannot observe inconsistent file system states. Currently, we are exploring methods that reuse the file system's locking protcols to ensure that eVFS operations can be committed atomically.

%Our plan is to acquire locks on file system objects and use operation logs~\cite{Pillai2017, Bhat2017} to simplify transaction aborts when lock acquisition fails (e.g., to avoid deadlock).

%Instead, our plan is to use an optimistic concurrency control scheme for committing eVFS operations in existing file systems. This approach will ensure that existing VFS applications are minimally affected by eVFS operations, even when eVFS applications may run long transactions. In turn, the eVFS applications must be designed to handle transaction aborts. Our implementation will take advantage of recent file system designs that use multiple operation logs, track dependencies between operations, and merge the operations.


%% Ashvin - To ensure compatibility with existing applications, the operations are performed atomically using a transactional interface.

% (jsun): which paper talks about running TM application with non-TM applications?

\vspace{-0.75em}
\section{Conclusion\label{sec:Conclusion}}
\vspace{-0.25em}

The eVFS interface exposes a new, low-level file system abstraction that enables control over allocation and modification of file system objects and the mappings between them. These operations are necessary for building generic file system management applications that make fine-grained updates to file system metadata. We showed the feasiblity of our approach by building a file system conversion tool. The application requires no changes to support a file system that implements the eVFS interface. We believe eVFS will enable exciting new applications and reduce the programming effort for building them.

% The eVFS interface exposes a new, low-level file system abstraction that enables control over allocation and modification of extents, inodes, and directory entries, and the mapping between extents and inodes. These operations are necessary for building generic file system management applications that perform fine-grained updates to file system metadata. We showed the feasiblity of our approach by building a file system conversion tool. The application requires no changes to support a file system that implements the eVFS interface. We believe eVFS will enable many exciting applications and reduce the programming effort for building these applications.

{\footnotesize \bibliographystyle{acm}
\bibliography{bibliography}}

\end{document}

