\section{Introduction\label{sec:Introduction}}
File system management applications help with maintaining, optimizing, and administering file systems. Examples of such applications include file-system upgrade tools, defragmentation tools, and file-system resizing tools. Unlike typical applications that are file-system agnostic because they use the virtual file-system interface (VFS) to access their data, the management applications perform low-level allocation, mapping, and placement of physical blocks in a file system. These operations are not exposed by the VFS API, and thus these applications must bypass the VFS, and access the file-system metadata directly.

As a result, file system developers and experts must write these applications from scratch for each file system, because they are tightly coupled with the format of the file system. For example, a defragmentation tool for Ext4 cannot be reused for Btrfs, not even in parts, because the two file systems use different formats for block allocation and free space management. The effort required for building these applications is significant, and thus newer file systems such as F2FS~\cite{lee2015f2fs} and BetrFS~\cite{betrfs-2015-tos} lack a rich set of management tools, which stymies their adoption and hinders innovation in file system technology.

The goal of our work is to simplify the development of file system management applications. The VFS interface has been highly successful because it abstracts the key objects (e.g., files and directories) and operations (e.g., create, delete, read, write) that are provided by any file system. Our approach is to provide a new abstraction, similar to VFS, that enables file system management applications to be written in a generic, file-system agnostic manner. Ideally, the applications are developed once using this interface and they work for file systems that implement this interface.  This relieves file system developers from the onus of building these essential (but often neglected) applications, and instead they can focus their effort on improving the file system itself.

We introduce the \textbf{E}xtended \textbf{V}irtual \textbf{F}ile \textbf{S}ystem (eVFS) interface, which provides a fine-grained abstraction for manipulating the file system. The key insight of eVFS is that the management applications operate on common abstractions that are shared across file systems, such as the allocation of file system objects (e.g., blocks or extents, inodes, and directory entries) and the mappings from logical blocks of a file to physical blocks.

We argue that these operations should be exposed to the management applications through a common API, and that doing so will simplify their design. The functionality required for a management task, such as defragmentation, is more easily expressed in terms of operations such as remapping extents when we abstract away the implementation details of those operations on a specific file system. We further argue that providing an API to manipulate the file system metadata does not change the file system's trust model. Management applications are already trusted to operate directly on metadata without the VFS, and bugs in them may cause file system inconsistency or corruption~\cite{Carreira2012,Gunawi08b}. Hence, exposing these operations through eVFS may improve the robustness of management applications, since the file-system specific implementation of the interface can be provided once by file system experts.

Ultimately, our aim is to expose these operations for online use without affecting existing file system applications that are unaware of management applications or the eVFS API. To do so, the eVFS API provides a transactional interface that ensures the atomicity of the eVFS operations. Currently, however, we have only explored offline use, where the transactional support provides crash consistency, which is often missing in management applications~\cite{gatla2018fsck}.

%The eVFS interface enables building applications that can be used across file systems. Consider a defragmentation tool that aims to reduce fragmentation in the file system. This application needs to find fragmented files and then relocate the fragmented blocks of the files into contiguous extents. Conventional defragmentation tools are coupled with the format of the file system. With the eVFS API, applications no longer need to operate on the low-level format of the file system. Instead, the API provides operations for allocating contiguous extents and remapping logical blocks of a file to a new extent. The decoupling of these ``subatomic'' file system operations from file system formats greatly simplifies the process of building these applications.

As proof of concept, we have built an offline file system conversion tool using the eVFS interface. This tool performs crash-consistent, in-place conversion of one file system to another entirely different file system. It can thus also be used to modify the file-system specific options of the file system, such as the file system size, or upgrade a file system (Ext3 to Ext4). The application is generic, and thus supporting additional file systems should require no modifications to the application. This experience suggests that the eVFS API will allow building a variety of generic, file system management applications.

%% Ashvin - To do so, this interface must provide precise control over the allocation and placement of file system resources, such as blocks and inodes. Fortunately, file systems internally operate on these same resources, but VFS currently does not expose these operations.

%These operations already exist for file systems that implements VFS. However, they are not currently exposed because they can lead to inconsistency or resource leak if used incorrectly. Therefore, we break down the VFS operations into components which form the building blocks for file system managemenet applications, and provide a transactional interface to ensure consistency of eVFS operations.

